% Chapter 6

\chapter{Alternative generalizations of the classical Cheeger constant}

\label{Chapter6}

In the introduction of this thesis, we defined the generalization of the classical Cheeger constant which was already introduced by Linial and Meshulam (see \cite{2}) and later independently by Gromov (see \cite{3}). But that notion is not the only way to generalize the classical Cheeger constant and we do not even know if it is the "best" way to generalize it, concerning the information these constants give about the stability of connectedness of simplicial complexes. In Chapter \ref{Chapter2} we saw that there is a close connection between the cosystolic norm of a cochain and the hitting number of a certain family of cycles, so we will use this connection to introduce other notions of the Cheeger constant which attain the same value as our earlier notion on a large family of complexes, but might be easier to determine in certain situations. These other notions were introduced by Kozlov (see \cite{13}).

\section{Alternative definitions of the higher dimensional Cheeger constants}

\begin{defi}
Let \(X\) be a simplicial complex and \(k\geq 1\). Furthermore, let \(\mathcal{F}\subset C_k(X)\) be a family of cycles, such that their supports are pairwise disjoint (i.e. \(\supp(F)\cap\supp(F')=\emptyset\) for all \(F,F'\in\mathcal{F}\), satisfying \(F\neq F'\)) and let
\[
P(\mathcal{F})\coloneqq\{C\subset X^{(k)}:|C\cap\supp(F)|=1\text{ for all }F\in\mathcal{F}\text{ and }C\subset\bigcup\limits_{F\in\mathcal{F}}\supp(F)\}
\]
denote the set of all possible choices of one simplex per cycle in the family \(\mathcal{F}\). Then we define
\[
\gamma_{\mathcal{F}}\coloneqq\frac{\min\limits_{c\in P(\mathcal{F})}\|\delta^k(c^*)\|}{|\mathcal{F}|}
\]
where \(c^*\) denotes the corresponding cochain to \(c\). The \(k\)-th \textbf{disjoint cycle expansion}\index{Disjoint cycle expansion} of \(X\) is then defined by:
\[
\gamma_k(X)\coloneqq\min\limits_{\mathcal{F}}\gamma_{\mathcal{F}}
\]
where the minimum is taken over all families of cycles \(\mathcal{F}\) as defined above.
\end{defi}

\begin{defi}
Let \(X\) be a simplicial complex and \(k\geq 1\). Furthermore, let \(\mathcal{F}\subset C_k(X)\) be an arbitrary family of cycles and let
\[
P'(\mathcal{F})\coloneqq\{C\subset X^{(k)}:|C\cap\supp(F)|\text{ is odd for all }F\in\mathcal{F}\}
\]
Then we define
\[
\rho_{\mathcal{F}}\coloneqq\frac{\min\limits_{c\in P'(\mathcal{F})}\|\delta^k(c^*)\|}{\tau(\mathcal{F})}
\]
where \(c^*\) denotes the corresponding cochain to \(c\) again. The \(k\)-th \textbf{hitting expansion}\index{Hitting expansion} of \(X\) is then defined by:
\[
\rho_k(X)\coloneqq\min\limits_{\mathcal{F}}\gamma_{\mathcal{F}}
\]
where the minimum is taken over all families of cycles \(\mathcal{F}\) as defined above.
\end{defi}

\begin{prop}\label{proposition411}
Let \(X\) be a simplicial complex and \(k\geq 1\), then we have:
\[
h_k(X)\leq\rho_k(X)\leq\gamma_k(X)
\]
\begin{proof}

\end{proof}
\end{prop}

\section{Relations to the Cheeger constant}

We will now show that the first disjoint cycle expansion (and thus also the first hitting expansion) of the simplex on \(n\) vertices equals the first Cheeger constant whenever \(n\) is not a power of \(2\).

\begin{thm}
Let $n$ not be a power of $2$, then we have:
\[
\gamma_1(\Delta^{[n]})=\rho_1(\Delta^{[n]})=\frac{n}{3}
\] 
\begin{proof}
Since $n$ is not a power of $2$ we can write it as $n=c(2t+1)$. Now consider the staircase graph $G_n(\lambda)$ given by the partition $\lambda=c\cdot\text{cor}(t)$. Since $G_n(\lambda)$ is bipartite we can partition the vertices of $G_n(\lambda)$ as $[n]=A\cup B\cup C$, with $A=\{v_1,\ldots,v_{ct}\}$, $B=\{w_1,\ldots,w_{ct}\}$ and $C=\{x_1,\ldots,x_c\}$, such that $C$ is the set of all isolated vertices and all edges of $G_n(\lambda)$ are contained in $E_{G_n(\lambda)}(A,B)$.\\
Construct a family of edge-disjoint cycles of the vertex set $[n]$ as follows:\\
For all edges $(v_i,w_j)$ satisfying $i+j\leq ct$, such that $(v_{ct-j+1},w_{ct-i+1})$ is not an edge in $G_n(\lambda)$ consider the cycle
\[
C_{ij}\coloneqq\{(v_i,w_j),(v_{ct-j+1},w_{ct-i+1}),(v_i,v_{ct-j+1}),(w_j,w_{ct-i+1})\}
\]
For all edges $e_{ij}=(v_i,w_j)$ satisfying $i+j\leq ct+1$, such that $e'_{ij}=(v_{ct-j+1},w_{ct-i+1})$ is also an edge in $G_n(\lambda)$ (for $i+j=ct+1$ they are equal), the set
\[
D\coloneqq\{e_{ij}, e'_{ij}:i+j\leq ct+1\text{, }e_{ij}\text{ and }e'_{ij}\text{ are edges in }G_n(\lambda)\}
\] can be partitioned into $t$ sets $B_1,\ldots,B_t$, each containing $c^2$ edges:
\[
B_k\coloneqq\{(v_i,w_j):(k-1)c+1\leq i\leq kc\text{, }c(t-k)+1\leq j\leq c(t-k+1)\}
\]
Now, each vertex from $A\cup B$ is only contained in edges from exactly one of the sets $B_k$. This means that for any $l=1,\ldots,c$ and any pair of edges $(v_i,w_j)\in B_{k_1}$ and $(v_{i'},w_{j'})\in B_{k_2}$ ($k_1\neq k_2$) the cycles $\{(v_i,w_j),(v_i,x_l),(w_j,x_l)\}$ and $\{(v_{i'},w_{j'}),(v_{i'},x_l),(w_{j'},x_l)\}$ are edge-disjoint. Furthermore, each set $B_k$ itself is a complete balanced bipartite graph (i.e. a graph in which each of the $c$ vertices from $A$ is adjecent to each of the $c$ vertices from $B$) so we can partition it into $c$ sets $B_k^1,\ldots,B_k^c$, such that all edges in $B_k^l$ are disjoint, for every $l=1,\ldots,c$. Thus, the cycles $\{(v_i,w_j),(v_i,x_l),(w_j,x_l)\}$ are edge-disjoint for all $(v_i,w_j)\in B_k^l$. The family of all these cycles united with the cycles $C_{ij}$ we defined before gives a family of edge-disjoint cycles, such that every edge of $G_n(\lambda)$ is contained in exactly one cycle and every cycle containes exactly one of the edges from $G_n(\lambda)$. Since the number of cycles in this family equals the number of edges in $G_n(\lambda)$ and we know that we have $h(G_n(\lambda))=\frac{n}{3}$ by \cite{1} (Theorem 4.2.) we get $\gamma_1(\Delta^{[n]})\leq\frac{n}{3}$ and by Proposition \ref{proposition411} we have $\gamma_1(\Delta^{[n]})=\rho_1(\Delta^{[n]})=\frac{n}{3}$.
\end{proof}
\end{thm}

The proof of the following statement can be found in \cite{6}.

\begin{thm}
Let $k+2$ devide $n$, then we have:
\[
\gamma_k(\Delta^{[n]})=\frac{n}{k+2}
\]
\end{thm}
