\begin{figure}[ht]
\centering
\begin{tikzpicture}
  \matrix (m) [matrix of math nodes, row sep=5em, column sep=1.5em]
    { \cdots & C^{k-1}(\Delta^{[n]},\mathbb{Z}_2) & C^{k}(\Delta^{[n]},\mathbb{Z}_2) & C^{k+1}(\Delta^{[n]},\mathbb{Z}_2) & \cdots \\
      \cdots & C^{k-1}(\Delta^{[n+1]},\mathbb{Z}_2) & C^{k}(\Delta^{[n+1]},\mathbb{Z}_2) & C^{k+1}(\Delta^{[n+1]},\mathbb{Z}_2) & \cdots \\ };
  { [start chain] \chainin (m-1-5);
    \chainin (m-1-4);
    { [start branch=A] \chainin (m-2-4)
        [join={node[right,labeled] {i}}];}
    \chainin (m-1-3) [join={node[above,labeled] {\partial}}];
    { [start branch=B] \chainin (m-2-3)
        [join={node[right,labeled] {i}}];}
    \chainin (m-1-2) [join={node[above,labeled] {\partial}}];
    { [start branch=C] \chainin (m-2-2)
        [join={node[right,labeled] {i}}];}
    \chainin (m-1-1); }
  { [start chain] \chainin (m-2-5);
    \chainin (m-2-4);
    \chainin (m-2-3) [join={node[above,labeled] {\partial}}];
    \chainin (m-2-2) [join={node[above,labeled] {\partial}}];
    \chainin (m-2-1); }
  { [start chain] \chainin (m-1-2);
  	\chainin (m-2-3) [join={node[above,labeled] {\varepsilon_1}}]; }
  { [start chain] \chainin (m-1-3);
  	\chainin (m-2-4) [join={node[above,labeled] {\varepsilon_1}}]; }
\end{tikzpicture}
  \caption{Chain complexes with coning maps}
  \label{figure2:Figure 2}
\end{figure}
