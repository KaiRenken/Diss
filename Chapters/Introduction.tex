% Introduction

\chapter{Introduction}

\label{Introduction}

The original and well-studied notion of the Cheeger constant of a graph can be considered as a measure of connectedness, measuring the relation of disconnecting relatively large connected components of a simple graph by removing a relatively small number of edges. If \(G=(V,E)\) is a simple graph (undirected and no loops or double edges allowed) on the vertex set \(V\) and the edge set \(E\subseteq\binom{V}{2}\) (where \(\binom{V}{2}\) denotes the set of all subsets of \(V\) of cardinality \(2\)), then the (\(0\)-th) Cheeger constant of \(G\) is defined as:
\[
h_0(G):=\min\left\{\frac{\left|\delta(A)\right|}{\left|A\right|}\text{ : }A\subset V\text{, }1\leq\left|A\right|\leq\frac{\left|V\right|}{2}\right\},
\]
with \(\delta(A):=\left\{e=(e_1,e_2)\in E\text{ : }e_1\in A\text{ and }e_2\in V\setminus A\right\}\).\\
