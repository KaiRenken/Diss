% Introduction

\chapter{Introduction}

\label{Introduction}

Let \(X\) be a simplicial complex and \(\varphi\in C^k(X)\), such that \(\|\delta^{k-1}(\phi)+\varphi\|\geq\|\varphi\|\) for every \(\phi\in C^{k-1}(X)\), where \(\delta^{k-1}\) denotes the coboundary map \(C^{k-1}(X)\rightarrow C^k(X)\), then we call \(\varphi\) a \textbf{cosystole}.\\
For general cochains \(\varphi\in C^k(X)\) we define the \textbf{cosystolic norm} of \(\varphi\) by:
\[
\|\varphi\|_{csy}:=\min\left\{\|\delta^{k-1}(\phi)+\varphi\|\text{ : }\phi\in C^{k-1}(X)\right\}
\]
Furthermore, any \(c\in C^k(X)\), satisfying \(c=\delta^{k-1}(\phi)+\varphi\) and \(\|c\|=\|\varphi\|_{csy}\) is called a \textbf{cosystolic form} of \(\varphi\).\\
The quotient
\[
\|\varphi\|_{exp}:=\frac{\|\delta^k(\varphi)\|}{\|\varphi\|_{csy}}
\]
is called the \textbf{coboundary expansion} of \(\varphi\) and
\[
h_k(X):=\min\limits_{\varphi\in C^k(X)\text{, }\delta^k(\varphi)\neq 0}\|\varphi\|_{exp}
\]
is called the \(k\)-th \textbf{Cheeger constant} of \(X\).\\
Furthermore, for any simplicial complex \(X\) and any \(1\leq k\leq dim(X)\), we define the following number (the largest norm, a cosystole can attain):
\[
C_{max}(X,k):=\max\left\{\|\varphi\|_{csy}\text{ : }\varphi\in C^k(X)\right\}
\]

